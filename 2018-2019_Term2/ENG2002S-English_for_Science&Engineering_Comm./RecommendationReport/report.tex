\documentclass[12pt]{article}
\usepackage{indentfirst}
\usepackage{siunitx}
\usepackage{graphicx}
\usepackage{subfigure}
\usepackage{float}
\usepackage{amsmath}
\usepackage{tikz}
\usepackage{pgfplots}
\usetikzlibrary{matrix}

\setlength{\parindent}{20pt}
\setlength{\oddsidemargin}{0.25cm}
\setlength{\evensidemargin}{0.25cm}
\setlength{\marginparsep}{0.5cm}
\setlength{\marginparwidth}{1.5cm}
\setlength{\textwidth}{160mm}
\renewcommand{\baselinestretch}{1.25}

\begin{document}
	\section{Introduction}
	Smart lamp-post is an intelligent system generally embedded in streetside lamp-posts, with a variety of functions integrated such as adaptive street lighting, PM2.5 detection, streetside WiFi(Access Point), video reconnaissance and so on. In order to control each lamp-post in the data center and collect data from sensors, a company from Hong Kong of which main business is manufacturing smart lamp-post need us to provide an approach to form a network between each lamp-post with both efficiency and reliability. Available options are Wi-Fi(wireless network based on IEEE 802.11), ZigBee(wireless network based on IEEE 802.15.4), 4G network and fiber. The difficulty of selecting among is that each option has its merits and precedents, which implies the recommendation will highly depend on the actual scenario. The purpose of this report is to evaluate each method and determine a method to form a hop-by-hop network which is most suitable for the actual scenario.
	
	\section{Selection Criterion}
	Based on demands of the company and the actual scenario, we filter the specification of each type of network into four selection criterion, which are latency, bandwidth, deployment cost and transmission cost. \\ \par 
	
	\textbf{Latency} \par 
	The first criterion is significant in a scenario while users are making video or voice call with others. 
	
	\section{Finding and Analysis}
	
	
	\section{Recommendation}
	
	
\end{document}