

\documentclass[11pt,conference,final,onecolumn]{IEEEtran}

\hyphenation{op-tical net-works semi-conduc-tor}
\usepackage{geometry}
\geometry{a4paper,scale=0.749}

\begin{document}

\title{
Profile Guided Source Coding in Compiler Optimization
}
%
%
% author names and IEEE memberships
% note positions of commas and nonbreaking spaces ( ~ ) LaTeX will not break
% a structure at a ~ so this keeps an author's name from being broken across
% two lines.
% use \thanks{} to gain access to the first footnote area
% a separate \thanks must be used for each paragraph as LaTeX2e's \thanks
% was not built to handle multiple paragraphs
%

\author{
\IEEEauthorblockN{
Chenhao Wu
}
\IEEEauthorblockA{
School of Science and Engineering\\
The Chinese University of Hong Kong, Shenzhen, China\\
}
}
% note the % following the last \IEEEmembership and also \thanks - 
% these prevent an unwanted space from occurring between the last author name
% and the end of the author line. i.e., if you had this:
% 
% \author{....lastname \thanks{...} \thanks{...} }
%                     ^------------^------------^----Do not want these spaces!
%
% a space would be appended to the last name and could cause every name on that
% line to be shifted left slightly. This is one of those "LaTeX things". For
% instance, "\textbf{A} \textbf{B}" will typeset as "A B" not "AB". To get
% "AB" then you have to do: "\textbf{A}\textbf{B}"
% \thanks is no different in this regard, so shield the last } of each \thanks
% that ends a line with a % and do not let a space in before the next \thanks.
% Spaces after \IEEEmembership other than the last one are OK (and needed) as
% you are supposed to have spaces between the names. For what it is worth,
% this is a minor point as most people would not even notice if the said evil
% space somehow managed to creep in.



% The paper headers
\markboth{Journal of \LaTeX\ Class Files,~Vol.~14, No.~8, August~2015}%
{Shell \MakeLowercase{\textit{et al.}}: Bare Demo of IEEEtran.cls for IEEE Journals}
% The only time the second header will appear is for the odd numbered pages
% after the title page when using the twoside option.
% 
% *** Note that you probably will NOT want to include the author's ***
% *** name in the headers of peer review papers.                   ***
% You can use \ifCLASSOPTIONpeerreview for conditional compilation here if
% you desire.




% If you want to put a publisher's ID mark on the page you can do it like
% this:
%\IEEEpubid{0000--0000/00\$00.00~\copyright~2015 IEEE}
% Remember, if you use this you must call \IEEEpubidadjcol in the second
% column for its text to clear the IEEEpubid mark.



% use for special paper notices
%\IEEEspecialpapernotice{(Invited Paper)}




% make the title area
\maketitle
%
%% As a general rule, do not put math, special symbols or citations
%% in the abstract or keywords.
%\begin{abstract}
%The abstract goes here.
%\end{abstract}
%
%% Note that keywords are not normally used for peerreview papers.
%\begin{IEEEkeywords}
%IEEE, IEEEtran, journal, \LaTeX, paper, template.
%\end{IEEEkeywords}






% For peer review papers, you can put extra information on the cover
% page as needed:
% \ifCLASSOPTIONpeerreview
% \begin{center} \bfseries EDICS Category: 3-BBND \end{center}
% \fi
%
% For peerreview papers, this IEEEtran command inserts a page break and
% creates the second title. It will be ignored for other modes.
\IEEEpeerreviewmaketitle
\setlength{\parindent}{20pt}


\section*{\bf Abstract}

In order to best utilize the run-time memory occupancy and to give powerful and intensive memory-consumption applications with higher compatibility on mobile and embedded systems, we proposed a compiler optimization, Profile Guided Source Coding (PGSC), which is led by the compilation process and aims to reduce the run-time memory occupancy of compiled programs. The optimization of PGSC is based on a profile generated by sample runs on the initially compiled executable, and the profile entails the access (load/read) frequencies on the allocated memory blocks performed by the program during run-time. Our strategy is during the compilation the memory accessing instructions of the program will be substituted by compress and decompress instructions, such that the memory blocks with lower frequency will be compressed into a compressed area when idled and will be decompressed into a run-time area when they are to be loaded. To achieve this objective we apply the Lempel-Ziv coding on the compression process and artificially selected several sample programs as the benchmark to examine the performance of programs compiled with PGSC optimization. The experiment shows that after applying PGSC optimization the average memory occupancy (AMO) and the maximum memory occupancy (MMO) of the identical program are decreased. Even though it would also bring an additional overhead in distributing the computing resources in run-time compression and decompression, the experiment shows this optimization is considerable in practice. In future, we will research on the combination of the PGSC optimization and the distributed cache system, and also to investigate the capability of combining PGSC optimization with contemporary channel coding schemes. 




%
%\begin{thebibliography}{1}
%
%\bibitem{IEEE1}
%S.~Yang, J.~Wang, Y.~Zhang, and Y.~Dong, ``On the Capacity Scalability of Line Networks with Buffer Size Constraints,'' in \textit{2019 IEEE International Symposium on Information Theory (ISIT)}, 2019.
%
%\bibitem{IEEE2}
%U.~Niesen, C.~Fragouli and D.~Tuninetti, ``On Capacity of Line Networks,'' in \textit{IEEE Transactions on Information Theory}, vol. 53, no. 11, pp. 4039-4058, Nov. 2007.
%
%\end{thebibliography}

\end{document}


