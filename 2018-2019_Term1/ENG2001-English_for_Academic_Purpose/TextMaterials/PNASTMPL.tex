%%%%%%%%%%%%%%%%%%%%%%%%%%%%%%%%%%%%%%%%%
% Proceedings of the National Academy of Sciences (PNAS)
% LaTeX Template
% Version 1.0 (19/5/13)
%
% This template has been downloaded from:
% http://www.LaTeXTemplates.com
%
% Original author:
% The PNAStwo class was created and is owned by PNAS:
% http://www.pnas.org/site/authors/LaTex.xhtml
% This template has been modified from the blank PNAS template to include
% examples of how to insert content and drastically change commenting. The
% structural integrity is maintained as in the original blank template.
%
% Original header:
%% PNAStmpl.tex
%% Template file to use for PNAS articles prepared in LaTeX
%% Version: Apr 14, 2008
%
%%%%%%%%%%%%%%%%%%%%%%%%%%%%%%%%%%%%%%%%%

%----------------------------------------------------------------------------------------
%	PACKAGES AND OTHER DOCUMENT CONFIGURATIONS
%----------------------------------------------------------------------------------------

%------------------------------------------------
% BASIC CLASS FILE
%------------------------------------------------

%% PNAStwo for two column articles is called by default.
%% Uncomment PNASone for single column articles. One column class
%% and style files are available upon request from pnas@nas.edu.

%\documentclass{pnasone}
\documentclass{pnastwo}

%------------------------------------------------
% POSITION OF TEXT
%------------------------------------------------

%% Changing position of text on physical page:
%% Since not all printers position
%% the printed page in the same place on the physical page,
%% you can change the position yourself here, if you need to:

% \advance\voffset -.5in % Minus dimension will raise the printed page on the 
                         %  physical page; positive dimension will lower it.

%% You may set the dimension to the size that you need.

%------------------------------------------------
% GRAPHICS STYLE FILE
%------------------------------------------------

%% Requires graphics style file (graphicx.sty), used for inserting
%% .eps/image files into LaTeX articles.
%% Note that inclusion of .eps files is for your reference only;
%% when submitting to PNAS please submit figures separately.

%% Type into the square brackets the name of the driver program 
%% that you are using. If you don't know, try dvips, which is the
%% most common PC driver, or textures for the Mac. These are the options:

% [dvips], [xdvi], [dvipdf], [dvipdfm], [dvipdfmx], [pdftex], [dvipsone],
% [dviwindo], [emtex], [dviwin], [pctexps], [pctexwin], [pctexhp], [pctex32],
% [truetex], [tcidvi], [vtex], [oztex], [textures], [xetex]

\usepackage{graphicx}
\usepackage{subfigure}
\usepackage{float}
\usepackage{subcaption}
\usepackage[font=small,labelfont=bf]{caption}
\renewcommand{\baselinestretch}{1.25}

%------------------------------------------------
% OPTIONAL POSTSCRIPT FONT FILES
%------------------------------------------------

%% PostScript font files: You may need to edit the PNASoneF.sty
%% or PNAStwoF.sty file to make the font names match those on your system. 
%% Alternatively, you can leave the font style file commands commented out
%% and typeset your article using the default Computer Modern 
%% fonts (recommended). If accepted, your article will be typeset
%% at PNAS using PostScript fonts.

% Choose PNASoneF for one column; PNAStwoF for two column:
%\usepackage{PNASoneF}
%\usepackage{PNAStwoF}

%------------------------------------------------
% ADDITIONAL OPTIONAL STYLE FILES
%------------------------------------------------

%% The AMS math files are commonly used to gain access to useful features
%% like extended math fonts and math commands.

\usepackage{amssymb,amsfonts,amsmath}


%------------------------------------------------
% OPTIONAL MACRO FILES
%------------------------------------------------

%% Insert self-defined macros here.
%% \newcommand definitions are recommended; \def definitions are supported

%\newcommand{\mfrac}[2]{\frac{\displaystyle #1}{\displaystyle #2}}
%\def\s{\sigma}

%------------------------------------------------
% DO NOT EDIT THIS SECTION
%------------------------------------------------

%% For PNAS Only:
%%\contributor{Submitted to Proceedings of the National Academy of Sciences of the United States of America}
%%\url{www.pnas.org/cgi/doi/10.1073/pnas.0709640104}
%%\copyrightyear{2008}
%%\issuedate{Issue Date}
%%\volume{Volume}
%%\issuenumber{Issue Number}

%----------------------------------------------------------------------------------------

\begin{document}

%----------------------------------------------------------------------------------------
%	TITLE AND AUTHORS
%----------------------------------------------------------------------------------------

\title{A Study of Relationship between Sleep Quality and Academic Performance} % For titles, only capitalize the first letter

%------------------------------------------------

%% Enter authors via the \author command.  
%% Use \affil to define affiliations.
%% (Leave no spaces between author name and \affil command)

%% Note that the \thanks{} command has been disabled in favor of
%% a generic, reserved space for PNAS publication.

%% \author{<author name>
%% \affil{<number>}{<Institution>}} One number for each institution.
%% The same number should be used for authors that
%% are affiliated with the same institution, after the first time
%% only the number is needed, ie, \affil{number}{text}, \affil{number}{}
%% Then, before last author ...
%% \and
%% \author{<author name>
%% \affil{<number>}{}}

%% For example, assuming Garcia and Sonnery are both affiliated with
%% Universidad de Murcia:
%% \author{Roberta Graff\affil{1}{University of Cambridge, Cambridge,
%% United Kingdom},
%% Javier de Ruiz Garcia\affil{2}{Universidad de Murcia, Bioquimica y Biologia
%% Molecular, Murcia, Spain}, \and Franklin Sonnery\affil{2}{}}

\author{Vito WU 117010285\affil{1}{The Chinese University of Hong Kong, Shenzhen}}

\contributor{}

%----------------------------------------------------------------------------------------

\maketitle % The \maketitle command is necessary to build the title page

\begin{article}

%----------------------------------------------------------------------------------------
%	ABSTRACT, KEYWORDS AND ABBREVIATIONS
%----------------------------------------------------------------------------------------

\begin{abstract}
The objective of this research is to investigate and explore the relationship between sleep quality and academic performance of students in CUHK(SZ). Based on 91 received questionnaires of self-reported GPA and PSQI index, our finding is that PSQI score has positive correlation with higher GPA, and negative correlation with lower GPA. Among seven components in PSQI index system, we also find that two components, subjective sleep quality and sleep latency have most significant impact on self-reported GPA. 
\end{abstract}

%------------------------------------------------

\keywords{Student Sleep Quality|Factors of Academic Performance|Correlation Analysis} % When adding keywords, separate each term with a straight line: |

%------------------------------------------------

%% Optional for entering abbreviations, separate the abbreviation from
%% its definition with a comma, separate each pair with a semicolon:
%% for example:
%% \abbreviations{SAM, self-assembled monolayer; OTS,
%% octadecyltrichlorosilane}

% \abbreviations{}


%----------------------------------------------------------------------------------------
%	PUBLICATION CONTENT
%----------------------------------------------------------------------------------------

%% The first letter of the article should be drop cap: \dropcap{} e.g.,
%\dropcap{I}n this article we study the evolution of ''almost-sharp'' fronts

\section{Introduction}

\dropcap{R}eports in recent years indicate that high school and college students are facing a perceptible decrease on their average sleep qualities and sleep duration, and it can be verified by observations and interviews of self-behavior from students in CUHK(SZ). With physiological researches showing that both sleep qualities and sleep duration correlate with one's cognitive functions, a reasonable inference can be established that the effect of sleep qualities and sleep duration might also intermediately affect one's academic performance by affecting cognitive functions.\par
A supporting report proposed by a study group from Netherlands claimed that, sleep qualities and sleep duration have slight affect on students' academic performance. Concluded from their report and previous reports they referred to, the common and unavoidable problem in the research is lack of appropriate approaches to measure the real academic performance of the respondents. Principally previous researchers applied self-reported GPA and parent-reported GPA in their research to represent the academic performance, which would explicitly cause publication bias while respondents concerning about the potential privacy leak. For purpose of hedging the subjective bias, even though in our research the adopted approach is still subject-reported GPA, we designed a cross-validation section. If the cross-validation we proposed can support the result we received from subject-reported GPA, the impact of publication bias would be reduced.\par 
Our research focuses on the correlation between the sleep qualities and academic performance, and therefore the questions we planned to investigate are 
\begin{itemize}
	\item The relationship between sleep qualities and academic performance
	\item The respective relationship between academic performance and each factors of sleep qualities
\end{itemize}


%------------------------------------------------

\section{Methodologies}

\subsection{Measurements of Sleep Quality}

~\ In our research we use a scientific and quantitive index to measure the sleep qualities of respondents, the Pittsburgh Sleep Quality Index (PSQI), which contains 19 self-rated questions. Questionnaire is divided into 7 components which represents distinct factors of sleep qualities. The score 

\subsection{Measurements of Academic Performance}
~\ \\
\textbf{1. self-reported GPA} \par We designed questionnaires including PSQI questions and GPA information, and published them online. After two weeks of publication we terminate the data collecting stage and move to data analysis.

 
\textbf{2. In-class evaluation in MAT2040 Lecture} \par Since every two week there is an in-class evaluation in Linear Algebra lecture this semester, that students will be asked to answer one question relative to lecture and assignments. We expect to evolve in-class evaluation as an cross-validate experiment and analyze the corrigendum of their submissions and PSQI score. 

\subsection{Correlation Analysis}
~\ Statistical correlation analysis was applied in our research. The procedure of correlation analysis is to calculate the correlation coefficients of two variables, self-reported GPA and PSQI score, using the formula
\begin{equation}
	\rho = \frac{E[(X - \mu_x)(Y - \mu_y)]}{\sigma_x\sigma_y}
\end{equation}
The correlation is represented by the correlation coefficient $\rho$, which has a range of $-1 \leq \rho \leq 1 $, where positive $\rho$ indicates positive correlation and negative $\rho$ indicates negative correlation. Closer the absolute value of $\rho$ to 1 corresponds to a stronger correlation between two variables.

\subsection{Participants}

~\ Our research was performed in The Chinese University of Hong Kong, Shenzhen, a young university in southern China. Different from majority of universities in mainland China, CUHK(SZ) inherits the education and evaluation system from CUHK. The study fee in CUHK(SZ) is 98500 CNY per year according to official announcement.

%------------------------------------------------

\section{Results}

\subsection{Received Data}
~\ \\
\textbf{1. Self-reported GPA and PSQI}
\\ From November 22nd to December 4th, a questionnaires covering self-reported GPA and PSQI was posted online, and 117 students in CUHK(SZ) participated in the research with 91 answer sheets being eventually received. Collected and clustered data are shown as following
\begin{figure}[htbp]
	\centering
	\subfigure{
		\includegraphics[width=220pt]{q1.png}
		\caption{Statistics of Self-reported GPA and PSQI}
	}
\end{figure}
\begin{figure}[htbp]
	\centering
	\subfigure{
		\includegraphics[width=230pt]{q2.png}
		\caption{Calculation of Correlation Coefficients between PSQI and Self-reported GPA}
	}
\end{figure}
\begin{figure}[htbp]
	\centering
	\subfigure{
		\includegraphics[width=220pt]{q3.png}
		\caption{Calculation of Correlation Coefficients between PSQI Components and Self-reported GPA}
	}
\end{figure}
 In November 27th, we handed out PSQI questionnaires in MAT-2040 in-class evaluation and collect the result are shown as following
\begin{figure}[htbp]
	\centering
	\subfigure{
		\includegraphics[width=220pt]{q4.png}
		\caption{Statistics of In-class Evaluation Corrigendum and PSQI}
	}
\end{figure}
\section{Discussion}
From Fig 1 we can observe that when PSQI score increases, there is a significant growth in the proportion of GPA interval 3.4 - 3.7 and 3.7 - 4.0, and a notable decrease in the proportion of GPA interval 2.7 - 3.0.\par 
After calculating the correlation coefficients for each GPA interval we obtain a table as Fig 2. From Fig 2 we can clearly conclude that PSQI score is positively correlated with higher self-reported GPA and negatively correlated with lower self-reported GPA, which indicates that better overall 
\end{article}
\end{document}